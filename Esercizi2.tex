\documentclass[a4paper]{article}

\makeatletter
\title{Istituzioni di Geometria}\let\Title\@title
\author{Andrea Gallese}\let\Author\@author
\date{\today}\let\Date\@date

\usepackage{Gallo}

\renewcommand{\P}{\mathbb{P}}

\begin{document}
	\Intitola
	
\begin{ex}[3.6]{Mostra che il fibrato tangente $ T K $ della bottiglia di Klein $ K $ ha una sezione mai nulla ma non ha un frame.}
	
	Otteniamo la bottiglia di Klein quozientando il suo rivestimento universale $ \R^2 $ per l'azione del gruppo di diffeomorfismi $ G = \langle (x+1, \, -y), \, (x, \, y+1) \rangle $. Il campo vettoriale costante $ (1,\, 0) $ è invariante per l'azione di $ G $ e passa pertanto al quoziente, fornendo la sezione mai nulla del tangente desiderata.\\
	
	Tuttavia, un frame non può esistere, perché questo indurrebbe un'orientazione sulla bottiglia, che abbiamo visto in classe non essere orientabile: sarebbe infatti sufficiente scegliere su ogni tangente $ T_pK $ l'orientazione del frame, perché questo garantisce la locale coerenza delle orientazioni. 
\end{ex}

\begin{ex}[3.7]{Mostra che il fibrato tangente $ TM $ di una varietà $ M $ è sempre orientabile, anche se $ M $ non lo è.}
	
	Osserviamo innanzitutto che il tangente in un punto del tangente $ (p, \, v) \in TM $ sarà della forma $ T_pM \oplus T_v(T_pM) $. Il secondo addendo è il tangente ad un vettore in uno spazio vettoriale e pertanto canonicamente identificato con lo spazio stesso: possiamo quindi scrivere
	$$ T_{(p, \, v)}(TM) = T_pM\oplus T_pM. $$
	
	Le carte del tangente sono, per costruzione, il pullback delle carte di $ M $ tramite la proiezione $ \pi\colon TM \to M $: ad ogni carta $ \varphi\colon U \to V $ in $ M $, corrisponde una carta di $ TM $ della forma
	\begin{align*}
		\varphi^*\colon \pi^{-1}(U) &\to V \times \R^n \\
		(p, \, v)& \mapsto (\varphi(p), \, d\varphi_p(v))	
	\end{align*}
	Il corrispondente differenziale sarà un isomorfismo $ d\varphi^*_{(p, v)} $ che agisce allo stesso modo sulle due componenti del tangente
	\[  d\varphi_p \oplus d\varphi_p \colon T_pM \oplus T_pM \to \R^n \oplus \R^n. \]
	
	L'orientazione di una somma diretta è indotta da quella degli addendi: giustapponendo due basi positive, otterremo una base positiva. Scelta comunque un'orientazione su ogni $ T_pM $, questa induce una scelta di orientazioni coerente sul tangente! Infatti che $ d\varphi_p $ preservi o meno l'orientazione, comunque $ d\varphi^*_{(p, v)} = d\varphi_p \oplus d\varphi_p $ la preserva: una base positiva di $ T_p(M) $ viene mandata nella giustapposizione di due basi positive oppure di due basi negative, in entrambi i casi una base positiva della somma diretta.
	

\end{ex}

\begin{ex}[3.9]{Costruisci un fibrato $ E \to K $ con fibra $ F = S^1 $
		sopra la bottiglia
		di Klein $ M $, tale che $ E $ sia una 3-varietà compatta orientabile (consiglio: puoi
		usare un esercizio della prima settimana).}
	
	Sia $ E $ la 3-varietà costruita nell'esercizio 1.8. Osserviamo che la proiezione $ \pi\colon \R^3 \to \R^2 $ sul piano $ y=0 $ è compatibile con l'azione del gruppo $ \Gamma $ su $ \R^3 $, nel senso che agire sull'intero spazio e poi proiettare sul piano è equivalente a restringere l'azione del gruppo al piano. Ovviamente, sul piano il gruppo è generato da solo $ \tilde{f}(x, z) = (x+1, z) $ e $ \tilde{h}(x, z) = (-x, z+1) $ e il quoziente è quindi una bottiglia di Klein. Otteniamo pertanto, passando al quoziente, una mappa suriettiva
	\[ p\colon E \to K. \]
	Per capire com'è fatta la fibra sopra un punto $ m \in K $, restringiamoci a considerare cosa succede sopra un quadrato fondamentale di $ \R^2 $: sul quadrato ci sarà esattamente un punto corrispondente ad $ m $, la cui contrimmagine tramite la proiezione ortogonale è la retta in $ \R^3 $ ortogonale al piano in quel punto ($ x = m_x, z = m_z $)! Su questa retta il gruppo $ \Gamma $ agisce per traslazione in modo discreto (tramite $ \tilde{g}(y) = (y+1) $), ne segue che la fibra sarà diffeomorfa a $ \S^1 $, come volevamo. Capito questo, la conclusione è immediata: prendendo un intorno $ U $ del punto (che possiamo assumere contenuto nel quadrato fondamentale), questo si solleverà al cilindro sovrastante, su cui $ \Gamma $ agisce per traslazione come prima! Il quoziente sarà pertanto diffeomorfo a $ U \times \S^1 $.
	
	\begin{center}
		\includegraphics[scale=0.5]{disegnino}
	\end{center}
\end{ex}

\begin{ex}[4.2]{Data una matrice quadrata $ A $, sia $ X_A $ il campo vettoriale su $ \R^n $ dato da $ X_A(x) = Ax $. Mostra che
		\[[X_A, X_B] = X_{BA-AB}.\]}
	I campi vettoriali in questione hanno per coordinate le colonne della rispettiva matrice
	\[ X_A = A^i \frac{\partial }{\partial x_i}, \]
	dunque possiamo scrivere esplicitamente le coordinate della parentesi di Lie
	\[ [X_A, X_B]
	= A^i \frac{\partial B^j }{\partial x_i} - B^i \frac{\partial A^j }{\partial x_i}
	= A^i B^{ij} \frac{\partial }{\partial x_i} - B^i A^{ij} \frac{\partial }{\partial x_i}
	= {\left(A^iB^{ij}-B^iA^{ij}\right)} \frac{\partial }{\partial x_i},  \]
	che è proprio la colonna $ i $-esima di $ BA-AB $!
\end{ex}

\begin{ex}[4.3]{Dimostra la identità di Jacobi: dati tre campi vettoriali $ X $, $ Y $, $ Z $ su una varietà $ M $, vale
\[[[X, Y ], Z] + [[Y, Z], X] + [[Z, X], Y ] \equiv 0.\]}
	Su ogni $ f \in C^\infty(M) $ il primo addendo del campo nella tesi agisce, per definizione (prima della parentesi esterna, poi di quella rimanente), come
	\begin{align*}
		[[X, Y ], Z](f) & = [X, Y]Z(f) - Z[X, Y](f) \\
		& = XYZ(f) - YXZ(f) - ZXY(f) + ZYX(f).
	\end{align*}
	Per ottenere gli altri addendi è sufficiente variare in modo ciclico i campo $ X, \, Y, \, Z $:
	\begin{align*}
	[[X, Y ], Z](f) & = XYZ(f) - YXZ(f) - ZXY(f) + ZYX(f), \\
	[[Z, X ], Y](f) & = ZXY(f) - XZY(f) - YZX(f) + YXZ(f), \\
	[[Y, Z ], X](f) & = YZX(f) - ZYX(f) - XYZ(f) + XZY(f).
	\end{align*}
	Rimane solo da osservare che ogni permutazione dei tre campi compare esattamente due volte, con segni discordi: segue che il campo risultante agisce su tutte le funzioni come il campo nullo e, pertanto, vi coincide.
\end{ex}

\begin{ex}[4.6]{Mostra che gli unici sottogruppi di Lie connessi di $ SO(3) $ sono l’identità, $ SO(3) $, e i sottogruppi isomorfi a $ \mathbb{S}^1 $ che descrivono le rotazioni intorno ad un asse.}
	
	Poiché i sottogruppi di Lie connessi di $ SO(3) $ sono in corrispondenza biunivoca con le sottoalgebre di Lie di $ \mathfrak{so}(3) = A(3) $ (l'algebra delle matrici antisimmetriche), è sufficiente classificare queste ultime. Dobbiamo quindi mostrare che le uniche sottoalgebre non banali sono quelle di dimensione 1 e che a queste corrispondono sottogruppi connessi isomorfi a $ \S^1 $.\\
	
	Tutti i sottospazi di dimensione 1 sono chiaramente sottoalgebre, infatti sono chiuse per le parentesi di Lie perché questa è bilineare: prese comunque $ \alpha X $ e $ \beta X $ nel sottospazio Span$ (X) $, abbiamo $$  [\alpha X, \beta X] = \alpha\beta[X, X] = 0 \in \text{Span}(X).  $$
	
	Viceversa, nessuno sottospazio di dimensione 2 è chiuso per le parentesi di Lie: è sufficiente osservare che 
	\begin{align*}
		\left(\begin{matrix}
		0 & -a & b \\
		a & 0 & -c \\
		-b & c & 0 
		\end{matrix}\right) \mapsto \left(\begin{matrix}
		a \\ b \\ c
		\end{matrix}\right)
	\end{align*}
	è un isomorfismo tra l'algebra di Lie delle matrici antisimmetriche $ \mathcal{A}(3) $ e $ \R^3 $ dotato del prodotto vettoriale. Quest'ultima algebra non ha sottoalgebre di dimensione 2, perché il prodotto vettoriale di due vettori linearmente indipendenti non appartiene al sottospazio generato da questi: $$  v \times w \notin \text{Span}(v, \, w).  $$
	
	Rimane da mostrare che alle sottoalgebre di dimensione 1 sono associati sottogruppi isomorfi a $ \mathbb{S}^1 $. A ogni matrice $ M \in \mathcal{A}(3) $ è associato il sottogruppo degli elementi della forma $ e^{tM} $ al variare di $ t \in \R $; ci aspettiamo che questo sia il gruppo di rotazioni attorno ad un asse fissato. Possiamo quindi affrontare il problema nella direzione opposta: tutti i sottogruppi di rotazione attorno ad un asse sono sottogruppi di Lie isomorfi ad $ \S^1 $ e, poiché ne troviamo uno per ogni direzione sul tangente (associando ad una matrice antisimmetrica il suo esponenziale), questi saranno tutti i sottogruppi di dimensione 1.
\end{ex}

\begin{ex}[5.2]{Mostra che due embedding $ f, g \colon \R \to \R^2 $ sono sempre isotopi.}
	
	Consideriamo degli intorni tubolari $ i_f, i_g\colon \R^2 \hookrightarrow \R^2 $ di, rispettivamente, $ f(\R) $ e $ g(\R) $. Questi sono due embedding di $ \R^2 $ in una varietà connessa e sono pertanto isotopi (a meno di precomporre $ i_g $ con una riflessione a scelta); ovvero esiste $$  F_t\colon \R^2 \times \R \to \R^2  $$ per cui $ F_0 = i_f $ e $ F_1 = i_g $ (oppure $ i_g \circ \rho $).
	
	Restringendo questa isotopia alla prima coordinata, otteniamo l'isotopia voluta: \[ \tilde{F_t}\colon \R_x \to \R^2 \]
	tale che $ \tilde{F_0} = f $ e $ \tilde{F_1} = g $ (scegliendo, nel caso, $ \rho $ la riflessione rispetto all'asse $ x $). 
	
\end{ex}

%\texttt{\begin{ex}[5.3]{Sia $ E \to M $ un fibrato. Mostra che una sottovarietà $ S \subset E $ è immagine di una sezione $ \Leftrightarrow $ interseca trasversalmente ogni fibra in un punto.}
%	
%	L'osservazione fondamentale è che, detta $ \pi\colon E \to M $ la proiezione del fibrato, il differenziale in ogni punto $ d\pi_p\colon T_pE \to T_{\pi(p)}M $ è una mappa lineare suriettiva di nucleo è $ \ker d\pi_p = T_pE_p $. Infatti, per definizione di fibrato, abbiamo un intorno $ U $ di $ \pi(p) $ per cui
%	\[ \begin{tikzcd}
%	\pi^{-1}(U) \dar["\pi"] \rar["\sim" above, "\varphi" below] & U \times F \arrow[dl, "\pi_U" ] \\
%	U
%	\end{tikzcd}, \]
%	che a livello di differenziali ci dice che $ d\pi_p =  d \pi_{\varphi(p)}\circ d\varphi_p $: visto che il primo è un isomorfismo, il nucleo sarà proprio il tangente della fibra.\\
%	
%	Ora la dimostrazione è una passeggiata.
%	Supponiamo di avere una sottovarietà
%	\[ \begin{tikzcd}
%	S \rar ["i", hook] \arrow[dr, "\psi" below, dashed]  & E \dar["\pi" right] \\
%	& M
%	\end{tikzcd} \]
%	che interseca trasversalmente ogni fibra in esattamente un punto. La composizione $ \psi = \pi \circ i $ è dunque bigettiva e un locale diffeomorfismo: infatti \[ d\psi_p = d\pi_{i(p)}\circ di_p \] e per ipotesi $ di_p(T_pS) + d_{i(p)}T_{i(p)}E_p $.
%	
%	
%	Supponiamo di avere una sezione
%	\[ \begin{tikzcd}
%	E \dar["\pi" left] \\
%	M \uar["s" right, bend right = 50, dashed]
%	\end{tikzcd} \]
%	e mostriamo che otteniamo una sottovarietà trasversa a ogni fibra in esattamente un punto.
%\end{ex}}

\begin{ex}[5.5]{Sia $ f \colon \S^1 \to \R^3 $ un nodo (cioè un embedding liscio). Mostra che esiste un piano affine $ P \subseteq \R^3 $ tale che $ \pi \circ f \colon \S^1 \to P $ sia un’immersione, dove $ \pi $ è la proiezione ortogonale su $ P $.}
	
	Indichiamo con $ \pi_v $ la proiezione sul piano normale al vettore $ v \in \R^3 $. Identificando tutti i tangenti di $ \R^3 $ con $ \R^3 $ il differenziale della proiezione è $ d\pi_v = \pi_v $. Dunque $ \pi_v f $ è un'immersione in $ p \in \S^1 $ se e solo se $$ 0 \neq d(\pi_v f)_p = \pi_v (df_p), $$
	ovvero se e solo se $ v $ e $ df_p(1) $ sono paralleli. Per trovare un'immersione è quindi sufficiente trovare una direzione $ v $ diversa da quella di ciascun tangente di $ f(\S^1) $ (pensato come sottospazio di $ \R^3 $) e proiettare lungo quella; ci basta quindi mostrare che esiste una direzione con questa proprietà. Questo è equivalente, una volta appurato che la retrazione di $ \R^3\setminus{0} $ su $ \S^2 $ è liscia, a mostrare che non ci sono mappe lisce suriettive $ \S^1 \to \S^2 $: esplicitamente, la mappa che associa ad ogni punto di $ \S^1 $ la direzione $ f'/ ||f|| $ è liscia. Ogni mappa $ \S^1 \to \S^2 $ deve però avere immagine nulla per il lemma di Sard: ogni punto di $ \S^1 $ è infatti valore critico, perché finisce in una varietà di dimensione maggiore.
	
\end{ex}

\begin{ex}[5.6]{Siano $ M $ e $ N $ due varietà connesse orientate senza bordo di dimensione $ n \geq 3 $. Mostra che
		$$  \pi_1(M \# N) \simeq \pi_1(M) \ast \pi_1(N)  $$
		dove $ \ast $ indica il prodotto libero di gruppi (cerca la definizione in rete se non la conosci).}
	
	Siano $ f\colon \R^n \hookrightarrow M $, $ g\colon \R^n \hookrightarrow N $ gli embedding per i quali è avvenuto l'incollamento. Applicando il Teorema di Van Kampen al ricoprimento costituito dai due aperti $ U = M \setminus f(B) $ e $ V = N\setminus g(B) $ di $ M \# N $ scopriamo che il gruppo fondamentale a cui siamo interessati è il pushout
	\[ \begin{tikzcd}
	\pi_1(U \cap V) \rar\dar\arrow[dr, phantom, "\ulcorner", very near end] & \pi_1(U) \dar \\
	\pi_1(V) \rar & \pi_1(M \# N).
	\end{tikzcd} \]
	Osserviamo che $ U \cap V $ è, per definizione di somma connessa, $ f(\R^n \setminus B) = g(\R^n \setminus B) $, quindi omeomorfo a $ \R^n \setminus B $, pertanto semplicemente connesso (visto che $ n \geq 3 $). Fondamentalmente per la stessa ragione $ U $ e $ M $ avranno lo stesso gruppo fondamentale, così come $ V $ e $ N $: è come togliere a $ \R^n $ la palla unitaria; applicando diligentemente il Teorema di Van Kampen al ricoprimento di $ M $ costituito dai due aperti $ M \setminus f(\frac{1}{2}\bar{B}) $ (che si retrae per deformazione su $ U $, che sfortunatamente non è aperto in $ M $) e $ f(B) $ otteniamo il seguente diagramma di pushout
	\[ \begin{tikzcd}
	\pi_1((M \setminus f(\frac{1}{2}\bar{B})) \cap f(B)) \rar\dar\arrow[dr, phantom, "\ulcorner", very near end] & \pi_1(M \setminus f(\frac{1}{2}\bar{B})) \dar \\
	\pi_1(f(B)) \rar & \pi_1(M).
	\end{tikzcd} \]
	I gruppi di sinistra sono entrambi banali, quello sotto evidentemente, quello sopra perché l'intersezione si retrae per deformazione su $ f(\S^{n-1}) $, che sappiamo essere semplicemente connesso per $ n\geq 3 $. Segue che la mappa di destra è un isomorfismo
	\[ \pi_1(U) \simeq \pi_1(M \setminus f(\frac{1}{2}\bar{B})) \to \pi_1(M). \]
	Possiamo dunque riscrivere il primo diagramma come
	\[ \begin{tikzcd}
	1 \rar\dar\arrow[dr, phantom, "\ulcorner", very near end] & \pi_1(M) \dar \\
	\pi_1(N) \rar & \pi_1(M \# N),
	\end{tikzcd} \]
	che è precisamente la tesi (perché il coprodotto in \textbf{Grp} è il prodotto libero).
	
\end{ex}

\end{document}