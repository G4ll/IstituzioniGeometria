\documentclass[a4paper]{article}

\makeatletter
\title{Algebraic Topology}\let\Title\@title
\author{Andrea Gallese}\let\Author\@author
\date{\today}\let\Date\@date

\usepackage{Gallo}

\begin{document}
	\Intitola

\begin{ex}[1]
	Consideriamo due atlanti costituiti dalle singole carte $$  \varphi(x) = x \qquad\text{ e }\qquad \psi(x) = \begin{cases}
	x \quad \text{ se } x \geq 0, \\
	2x \quad \text{se } x < 0.
	\end{cases}  $$
	Questi sono atlanti banalmente: le carte sono chiaramente omeomorfismi ed essendo uniche non c'è altro da verificare. I due atlanti non sono però compatibili: infatti la funzione di transizione $ \psi \varphi^{-1} $ è continua ma non liscia. La funzione $ \psi^{-1} $ tuttavia, letta in carte, coincide con l'identità su $ \R $ ed è pertanto un diffeomorfismo.
\end{ex}

\begin{ex}[2]
	L'esistenza della struttura liscia si ottiene come al solito facendo salire la struttura: ogni carta $ \varphi\colon U \to \R^n $ di $ N $ si solleva a una carta $ \varphi \circ f \colon f^{-1}(U) \to \R^n $; queste sono omoeomorfismi perché composizione di omeomorfismi e si incollano perché il sollevamento è coerente: $$  (\varphi_i \circ f) \circ (\varphi_j \circ f)^{-1} = \varphi_i \circ \varphi_j^{-1}, $$
	che sappiamo essere lisce per ipotesi. \\
	
	Occupiamoci ora dell'unicità. Supponiamo di aver dato due diverse strutture lisce $ M $, $ M' $. La composizione di diffeomorfismi rimane tale, dunque l'identità
	\[\begin{tikzcd}
	id_M \colon M \rar["f"] & N \rar["f^{-1}"] & M
	\end{tikzcd}  \]
	è liscia in entrambe le direzioni! Ma chiedere che l'identità sia liscia è proprio equivalente a chiedere che le mappe di transizione tra le carte lo siano: $ \varphi \circ id_M \circ \psi^{-1} = \varphi\circ\psi^{-1} $ per ogni coppia di carte $ (\varphi, \, \psi) $ in atlanti diversi, ovvero che i due atlanti siano compatibili.
\end{ex}

\begin{ex}[4]
	Osserviamo che $ dp_z= p'(z) $, dunque che $ p $ è un locale diffeomorfismo su tutti i punti in cui la derivata non si annulla, ossia su $ \C \setminus S $. La controimmagine di punto $ w \in \C \setminus p(S) $ conta esattamente $ d $ punti, infatti questi sono gli zeri di $ q(z) = p(z)-w $, che non può avere radici doppie perché la sua derivata $ q' = p' $ si annulla solo in $ p^{-1}(p(S)) $ e che deve avere $ d $ radici per il teorema fondamentale dell'algebra. \\
	
	Da qui la conclusione è puramente topologica: prendiamo $ \varphi_i\colon U_i \to W_i $ locali diffeomorfismi da intorni delle $ d $ radici di $ q $ ad altrettanti intorni di $ w $, intersechiamo questi intorni per ottenere un unico $ W = \bigcap_{i = 1}^d W_i $. Restringendo i diffeomorfismi di sopra otteniamo diffeomorfismi $ \varphi_i\colon \varphi_i^{-1}(W) \to W $, mostrando in particolare che $ W $ è un aperto banalizzante di $ w $; concludiamo per arbitrarietà del punto $ w $.
\end{ex}

\end{document}